% This is based on "sig-alternate.tex" V1.9 April 2009
% This file should be compiled with V2.4 of "sig-alternate.cls" April 2009
%
\documentclass{report}

\usepackage[english]{babel}
\usepackage{graphicx}
\usepackage{tabularx}
\usepackage{subfigure}
\usepackage{enumitem}
\usepackage{url}

\usepackage{color}
\definecolor{orange}{rgb}{1,0.5,0}
\definecolor{lightgray}{rgb}{.9,.9,.9}
\definecolor{java_keyword}{rgb}{0.37, 0.08, 0.25}
\definecolor{java_string}{rgb}{0.06, 0.10, 0.98}
\definecolor{java_comment}{rgb}{0.12, 0.38, 0.18}
\definecolor{java_doc}{rgb}{0.25,0.35,0.75}

% code listings

\usepackage{listings}
\lstloadlanguages{Java}
\lstset{
	language=Java,
	basicstyle=\scriptsize\ttfamily,
	backgroundcolor=\color{lightgray},
	keywordstyle=\color{java_keyword}\bfseries,
	stringstyle=\color{java_string},
	commentstyle=\color{java_comment},
	morecomment=[s][\color{java_doc}]{/**}{*/},
	tabsize=2,
	showtabs=false,
	extendedchars=true,
	showstringspaces=false,
	showspaces=false,
	breaklines=true,
	numbers=left,
	numberstyle=\tiny,
	numbersep=6pt,
	xleftmargin=3pt,
	xrightmargin=3pt,
	framexleftmargin=3pt,
	framexrightmargin=3pt,
	captionpos=b
}

% Disable single lines at the start of a paragraph (Schusterjungen)

\clubpenalty = 10000

% Disable single lines at the end of a paragraph (Hurenkinder)

\widowpenalty = 10000
\displaywidowpenalty = 10000
 
% allows for colored, easy-to-find todos

\newcommand{\todo}[1]{\textsf{\textbf{\textcolor{orange}{[[#1]]}}}}

% consistent references: use these instead of \label and \ref

\newcommand{\lsec}[1]{\label{sec:#1}}
\newcommand{\lssec}[1]{\label{ssec:#1}}
\newcommand{\lfig}[1]{\label{fig:#1}}
\newcommand{\ltab}[1]{\label{tab:#1}}
\newcommand{\rsec}[1]{Section~\ref{sec:#1}}
\newcommand{\rssec}[1]{Section~\ref{ssec:#1}}
\newcommand{\rfig}[1]{Figure~\ref{fig:#1}}
\newcommand{\rtab}[1]{Table~\ref{tab:#1}}
\newcommand{\rlst}[1]{Listing~\ref{#1}}

% General information

\title{Distributed Systems -- Assignment 2}

% Use the \alignauthor commands to handle the names
% and affiliations for an 'aesthetic maximum' of six authors.

\numberofauthors{3} %  in this sample file, there are a *total*
% of EIGHT authors. SIX appear on the 'first-page' (for formatting
% reasons) and the remaining two appear in the \additionalauthors section.
%
\author{
% You can go ahead and credit any number of authors here,
% e.g. one 'row of three' or two rows (consisting of one row of three
% and a second row of one, two or three).
%
% The command \alignauthor (no curly braces needed) should
% precede each author name, affiliation/snail-mail address and
% e-mail address. Additionally, tag each line of
% affiliation/address with \affaddr, and tag the
% e-mail address with \email.
%
% 1st. author
\alignauthor Student One\\
	\affaddr{ETH ID XX-XXX-XXX}\\
	\email{one@student.ethz.ch}
% 2nd. author
\alignauthor Student Two\\
	\affaddr{ETH ID XX-XXX-XXX}\\
	\email{two@student.ethz.ch}
%% 3rd. author
\alignauthor Student Three\\
	\affaddr{ETH ID XX-XXX-XXX}\\
	\email{three@student.ethz.ch}
%\and  % use '\and' if you need 'another row' of author names
%% 4th. author
%\alignauthor Student Four\\
% 	\affaddr{ETH ID XX-XXX-XXX}\\
% 	\email{four@student.ethz.ch}
%% 5th. author
%\alignauthor Student Five\\
% 	\affaddr{ETH ID XX-XXX-XXX}\\
% 	\email{five@student.ethz.ch}
%% 6th. author
%\alignauthor Student Six\\
% 	\affaddr{ETH ID XX-XXX-XXX}\\
% 	\email{six@student.ethz.ch}
}


\begin{document}

\maketitle

\begin{abstract}
Concisely state (i) which Android device you used, (ii) which tasks you completed and which are working correctly or limited, and (iii) what your specific enhancements are.
\end{abstract}

\section{Introduction}

Use the introduction for background information on the assignment.
See your assignment sheet for specific questions on the topic that you have to answer in this section.
Use references such as books \cite{hello}, papers and theses \cite{REST}, or specifications \cite{RFC2616} whenever available.
Web sites for documentation \cite{devServices}, tutorials, etc. are a special case.
In a thesis, you would put them as footnotes. At this stage, however, you will only have a few ``real references,'' so we put the Web sites into the bibliography.
Cite every source you used throughout the assignment.

For Assignment 2, please give a short overview of Web Services, especially their benefits.
Compare RESTful against WS\hbox{-}* Web Services by listing some advantages and disadvantages of both concepts.

\section{RESTful Web Services}
\begin{itemize}
	\item Describe shortly, how you designed your application to implement this Task. Which Android core elements did you use (e.g., Activity, Service, AsyncTask, Intent)?
	\item Explain why it is beneficial (or even required, since Android 3.0), to off-load networking tasks to AsyncTask. \textbf{Hint:} Explain what \emph{blocking}-methods are in the context of handling I/O Streams.
	\item Explain how you make use of the different methods provided by AsyncTask, such as \texttt{doInBackground}, \texttt{onPreExecute}, \texttt{onProgressUpdate}, \texttt{publishProgress} or \texttt{onPostExecute}.
\end{itemize}

\section{WS-* Web Services}
\begin{itemize}
	\item Describe shortly, how you designed your application to implement this Task. Did you reuse elements from Task 1?
	\item What are the roles of WSDL files and SOAP requests in WS-* Web Services?
	\item Explain why SOAP messages are exchanged in XML format and what unmarshalling to platform specific objects means.
\end{itemize}

\section{Cloud Services}
\begin{itemize}
	\item Which diagram type did you choose? You can show a screen shot and describe your custom functionalities, if you have implemented any.
\end{itemize}

\section{Your Phone As a Server}
\begin{itemize}
	\item Describe shortly, how you designed your application to implement this Task. Which Android core elements did you use this time?
	\item Show in a control flow diagram, pseudo code or your actual Java implementation, how you (would) do the handling of the connections on the server side. If you weren't able to implement the multi threading, explain it for a single threaded version.
	\item Did you implement sensing and/or actuation? How did you make use of the different HTTP methods, e.g. GET or PUT?
\end{itemize}

\section{Enhancements}

This is usually the final part of an assignment.
Here you are free to describe what you did, however, stick to a concise, scientific writing style.

\section{Conclusion}

Give an overall conclusion that summarizes the main challenges you encountered and your lessons learned.

% The following two commands are all you need in the
% initial runs of your .tex file to
% produce the bibliography for the citations in your paper.
\bibliographystyle{abbrv}
\bibliography{report}  % sigproc.bib is the name of the Bibliography in this case
% You must have a proper ".bib" file

%\balancecolumns % GM June 2007

\end{document}
